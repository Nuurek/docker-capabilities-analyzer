\chapter{Zakończenie}

\section{Zebrane doświadczenia}

W trakcie realizacji pracy autor zdobył liczne doświadczenia wynikające z~analizy, projektowania i~implementacji programu. Dogłębna analiza całego ekosystemu Dockera pozwoliła na pozyskanie wiedzy przydatnej nie tylko z~kwestii widzenia bezpieczeństwa, ale także ogólnie pojętego, zaawansowanego wykorzystania narzędzia Docker. Ponadto, projektowanie analizatora wymagało spędzenia wielu godzin na analizie kodu samego jądra Linuxa co pozwoliło na zapoznanie się technikami programistycznymi wykorzystanymi w~tak zaawansowanym projekcie. Ostatecznie, implementacja programu zakładała połączenie w~całość kilku projektów i~języków projektowania. Doprowadziło do potrzeby zapoznania się z~wieloma dokumentacjami, sięgającymi nawet 30 lat wstecz.

\section{Podsumowanie}

Popularność Dockera, tam samo jak i~innych znanych projektów informatycznych, sprawiła, że stał się on częstym celem ataków. Docker pozyskał również wielu użytkowników nieświadomych zagrożeń wynikających z~nieprawidłowego wykorzystania tego narzędzia. Istniejące wsparcie ze strony twórców wszystkich głównych części ekosystemu Dockera umożliwia naprawę błędów i~ciągły rozwój zabezpieczeń. Efekty konteneryzacji wybiegają jednak poza zamknięty świat usług Dockera, co powoduje istnienie poważnych wektorów ataku w~postaci serwisów zewnętrznych. Jednakże, jak pokazała analiza, większość istniejących luk jest zależna w~dużym stopniu od błędu ludzkiego, a~kultura szybkiego wdrażania oprogramowania, oparta na filozofii DevOps, jeszcze bardziej uwypukla te wady.
