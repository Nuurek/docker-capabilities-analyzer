\chapter{Realizacja aplikacji}

Analiza podatności pokazała, że większość wektorów ataku na ekosystem Dockera pochodzi z błędów konfiguracji demona i kontenerów Dockera, a także z błędów ludzkich związanych z wielousługową metodą wdrażana oprogramowania. Ogromne wsparcie od twórców Dockera pozwala na szybkie łatanie podatności środowiska. Jednakże aplikowanie aktualizacji bezpieczeństwa nadal jest zależne od użytkownika końcowego.

Błędy konfiguracyjne mogą pojawić się na wielu etapach pracy z ekosystemem Dockera. Pomijając zewnętrzne zależności można wyróżnić takie elementy środowiska jak pliki konfiguracyjne demona Dockera, uprawnienia powiązanych z nim plików i katalogów systemu gospodarza oraz opcje przekazane przez użytkownika w trakcie uruchomienia demona. Ponadto, konfiguracja kontenerów jest podatna na błędy wynikające z wykorzystania nieodpowiednich dyrektyw w plikach Dockerfile lub opcji użytych w trakcie uruchamiania kontenerów. Polityka bezpieczeństwa powinna brać pod uwagę scenariusz w jakim jest aplikowana i być odpowiednio dostosowywana w celu osiągnięcia zamierzonych celów.

Popularność Dockera dała wielu programistu dostęp do bardzo złożonego narzędzia jakim są kontenery. Jednakże, nie każdy użytkownik Dockera jest specjalistą od spraw bezpieczeństwa systemów komputerowych. Wiąże się to z tworzeniem podatności wynikających z nieznajomości pewnych aspektów wykorzystywanej technologii. Celem twórców wszystkich elementów ekosystemu Dockera powinno być wsparcie użytkowników w kwestiach zwiększania bezpieczeństwa tworzonych przez nich aplikacji kontenerowych. 

Wspomniany w poprzednich rozdziałach Docker Benchmark \cite{CISDockerBenchmark} został przeanalizowany przez twórców Dockera w celu stworzenia narzędzia weryfikującego wykorzystanie zalecanych dobrych praktyk. Wynikiem tej pracy jest implementacja skryptu [TODO]. W sposób statyczny analizuje on konfigurację demona i kontenerów Dockera sprawdzając spełnienie zaleceń oznaczonych w Docker Benchmark jako "Scored" (88/115 wszystkich zaleceń).

W ramach pracy zaprojektowano aplikację wspomagającą zautomatyzowaną analizę kontenerów Docker pod kątem wykrywania przydziału zbyt dużej ilości uprawnień. W porównaniu do skryptu [TODO] aplikacja analizuje wykorzystanie uprawnień w trakcie działania kontenera. Wynikiem uruchomienia aplikacji jest raport przedstawiający porównanie zadeklarowanych w konfiguracji uprawnień kontenera z rzeczywiście użytymi. Tym samym, daje użytkownikowi możliwości zmniejszenia wektora ataku, którym są nadmiarowe uprawnienia. W kolejnych podrozdziałach aplikacja jest zwana również analizatorem.

\section{Wykorzystane narzędzia}

\subsection{Berkeley Packet Filter}

\subsection{extended Berkeley Packet Filter}

\subsection{BPF Compiler Collection}

Berkeley Packet Filter Compiler Collection, w skrócie BCC, jest zbiorem narzędzi pozwalających na tworzenie programów śledzących i manipulujących wykorzystanie jądra Linuxa. BCC w znacznym stopniu ułatwia pisanie programów wykorzystujących BPF poprzez opakowanie kodu bajtowego BPF wewnątrz programów napisanych w języku C oraz udostępnienie interfejsów programistycznych w językach Lua i Python. Dodatkowo, repozytorium BCC zawiera dużą ilość przykładów i wbudowanych narzędzi (ponad 150) ułatwiających nowym programistom rozpoczęcie tworzenia własnych programów.

BCC dodatkowo analizuje kod programów napisanych w C dzięki czemu gwarantuje, że kod bajtowy będzie działał poprawnie. To narzędzie ma na celu zapewnienie interfejsu, który pozwali na tworzenie tylko i wyłącznie poprawnych programów BPF, zachowując równocześnie dostęp do wszystkich funkcjonalności. Ponadto, minimalizuje czas spędzony na przygotowaniu i kompilacji kod bajtowego BPF, umożliwiając skupienie się na tworzeniu docelowej aplikacji \cite{BCCReadme}.

\subsection{Python3.7}

\subsection{Docker SDK for Python}

\section{Schemat działania}

\section{Implementacja analizy w trakcie działania kontenera}

strace, go-audit, vagrant, analizator

\section{Wyniki przypadkowej analizy}

simple REST API or static nginx
